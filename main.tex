\documentclass[12pt, a4paper, oneside]{ctexart}
\usepackage{amsmath, amsthm, amssymb, appendix, bm, graphicx, subcaption, 
hyperref, mathrsfs}
\usepackage[a4paper,left=3cm,right=2cm,top=2.5cm,bottom=2.5cm]{geometry}

\linespread{1.5}
\newtheorem{theorem}{定理}[section]
\newtheorem{definition}[theorem]{定义}
\newtheorem{lemma}[theorem]{引理}
\newtheorem{corollary}[theorem]{推论}
\newtheorem{example}[theorem]{例}
\newtheorem{proposition}[theorem]{命题}
\renewcommand{\abstractname}{\Large\textbf{摘要}}

%\newcommand{\ulfrule}{\xleaders\hbox{\underline{ }}\hfill\kern0pt}

\usepackage{ulem}

% 生成一定长度下划线并在其填充文字的新命令
\newcommand{\wideunderline}[2][2em]{%
	\uline{\makebox[\ifdim\width>#1\width\else#1\fi][c]{#2}}%
}

\begin{document}
	
	\thispagestyle{empty}
	
	\begin{figure}[t!]
		\centering
		% 插入第一张子图
		\begin{subfigure}{1.\linewidth}
			\centering
			\includegraphics[width=0.9\linewidth]{assets/Icons/SJTU_1.png}
			%\caption{A red flower.}
		\end{subfigure}
		\vspace{0.1em}%\vfill
		
		\begin{subfigure}{1.\linewidth}
			\centering
			\includegraphics[width=0.9\linewidth]{assets/Icons/SJTU_2.png}
			%\caption{A red flower.}
		\end{subfigure}
		\vspace{0.1em}%\vfill
		
		% 插入第三张子图
		\begin{subfigure}{.5\linewidth}
			\centering
			\includegraphics[width=.7\linewidth]{assets/Icons/SJTU_3.png}
			%\caption{A yellow flower.}
		\end{subfigure}
	
		%\caption{Three flowers with different colors.}
		%\label{fig:fig1}
	\end{figure}
	
	%\vspace{1.5cm}
	%\vfill
	
	\begin{center}
		\Huge\textbf{X X X X 课 程}
	\end{center}
	
	\begin{center}
		\Huge\textbf{课程报告}
	\end{center}
	%\vspace{0.5cm}
	%\vfill
	\vspace{1cm}
	\begin{table}[h]
		\centering
		\Large
		\begin{tabular}{lc}
			\textbf{姓名:} & \wideunderline[6cm]{杨伸炉}  \\
			%\hrulefill{杨伸炉} \
			%\ %\underline{杨伸炉}
			%\ulfrule \\
			\textbf{学号:} & \wideunderline[6cm]{021050910071}  \\
			%\underline{021050910071}\ulfrule \\
			\textbf{学院:} & \wideunderline[6cm]{材料科学与工程学院}  \\
			%\underline{材料科学与工程学院}\ulfrule \\
			%\textbf{时间:} & 2021年7月 \\
		\end{tabular}
	\end{table}
	\vspace{0.5cm}
	
	\begin{center}
		\Large\textbf{2022年04月}
	\end{center}
	
	\newpage
	
	\thispagestyle{empty}
	\begin{abstract}
		这里是摘要. 
		\par\textbf{关键词:}这里是关键词; 这里是关键词. 
	\end{abstract}
	
	\newpage
	\pagenumbering{Roman}
	\setcounter{page}{1}

	\tableofcontents

	\newpage
	\setcounter{page}{1}
	\pagenumbering{arabic}
	
	\section{一级标题}
	
	\subsection{二级标题}
	
	
	这里是正文. 
	
	\newpage
	
	\begin{thebibliography}{99}
		\bibitem{a}作者. \emph{文献}[M]. 地点:出版社,年份.
		\bibitem{b}作者. \emph{文献}[M]. 地点:出版社,年份.
	\end{thebibliography}
	
	\newpage
	
	\begin{appendices}
		\renewcommand{\thesection}{\Alph{section}}
		\section{附录标题}
		这里是附录. 
	\end{appendices}
	
\end{document}
