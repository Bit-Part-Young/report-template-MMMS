\documentclass[12pt,a4paper,oneside]{ctexart}
\usepackage[a4paper,left=3cm,right=2cm,top=2.5cm,bottom=2.5cm]{geometry}
\usepackage{amsmath,amsthm,amssymb}
\usepackage{appendix}
\usepackage{bm}
\usepackage{graphicx}
\usepackage{subcaption}
\usepackage{hyperref}
\usepackage{mathrsfs}
\usepackage{ulem}

\linespread{1.5}


%---------------------------------------------------------------------
%   生成一定长度下划线并在其填充文字的新命令
%---------------------------------------------------------------------
\newcommand{\wideunderline}[2][2em]{%
	\uline{\makebox[\ifdim\width>#1\width\else#1\fi][c]{#2}}%
}

\begin{document}
	
	%---------------------------------------------------------------------
    %   封面页设置
    %---------------------------------------------------------------------
	% 首页无页眉页脚
	\thispagestyle{empty}
	
	% 学校图标排版
	\begin{figure}[!t]
		\centering
		
		\begin{subfigure}{1.\linewidth}
			\centering
			\includegraphics[width=0.9\linewidth]{assets/Icons/banner_zh_sjtu.png}
		\end{subfigure}
		
		\vspace{0.1em}%\vfill
		
		\begin{subfigure}{1.\linewidth}
			\centering
			\includegraphics[width=0.9\linewidth]{assets/Icons/banner_eng_sjtu.png}
		\end{subfigure}
		
		\vspace{0.1em}%\vfill
		
		\begin{subfigure}{.5\linewidth}
			\centering
			\includegraphics[width=.7\linewidth]{assets/Icons/badge_sjtu.png}
		\end{subfigure}
		
	\end{figure}
	
    %---------------------------------------------------------------------
	
	% 课程、个人信息
	\begin{center}
		\Huge\textbf{多尺度材料模拟与计算}
	\end{center}
	
	\begin{center}
		\LARGE\textbf{实验报告 - 简谐振子}
	\end{center}
	
	\vspace{1cm}
	
	\begin{table}[!h]
		\centering
		\Large
		\begin{tabular}{lc}
			\textbf{姓\qquad 名:} & \wideunderline[7cm]{杨伸炉}  \\
			\textbf{学\qquad 号:} & \wideunderline[7cm]{021050910071}  \\
			\textbf{学\qquad 院:} & \wideunderline[7cm]{材料科学与工程学院}  \\
		\end{tabular}
	\end{table}
	
	\vspace{0.5cm}
	
	\begin{center}
		\Large\textbf{2023年10月01日}
	\end{center}
	
	\newpage
	
	%---------------------------------------------------------------------
    %   摘要页
    %---------------------------------------------------------------------
	\thispagestyle{empty}
	\begin{abstract}
		这里是摘要. 
		\par\textbf{关键词:}这里是关键词; 这里是关键词. 
	\end{abstract}
	
	\newpage
	
	%---------------------------------------------------------------------
    %   目录页设置
    %---------------------------------------------------------------------
	\pagenumbering{Roman}
	\setcounter{page}{1}

	\tableofcontents

	\newpage
	
	%---------------------------------------------------------------------
    %   正文
    %---------------------------------------------------------------------
	\setcounter{page}{1}
	\pagenumbering{arabic}
	
	\section{一级标题}
	
	\subsection{二级标题}
	
	
	这里是正文. 
	
	\newpage
	
	%---------------------------------------------------------------------
    %   参考文献
    %---------------------------------------------------------------------
	\begin{thebibliography}{99}
		\bibitem{a}作者. \emph{文献}[M]. 地点:出版社,年份.
		\bibitem{b}作者. \emph{文献}[M]. 地点:出版社,年份.
	\end{thebibliography}
	
	\newpage
	
	%---------------------------------------------------------------------
    %   附录
    %---------------------------------------------------------------------
	\begin{appendices}
		\renewcommand{\thesection}{\Alph{section}}
		\section{附录标题}
		这里是附录. 
	\end{appendices}
	
\end{document}
