\documentclass[12pt,a4paper,oneside]{ctexart}
% \documentclass[a4paper]{article}
% \usepackage[margin=1.25in]{geometry}
\usepackage[a4paper,left=3cm,right=2cm,top=2.5cm,bottom=2.5cm]{geometry}
% \usepackage[inner=2.0cm,outer=2.0cm,top=2.5cm,bottom=2.5cm]{geometry}
% \usepackage{ctex}
\usepackage{color}
\usepackage{graphicx}
\usepackage{amssymb}
\usepackage{amsmath}
\usepackage{amsthm}
\usepackage{bm}
\usepackage{hyperref}
\usepackage{multirow}
\usepackage{mathtools}
\usepackage{enumerate}

\newcommand{\homework}[5]{
    \pagestyle{myheadings}
    \thispagestyle{plain}
    \newpage
    \setcounter{page}{1}
    \noindent
    \begin{center}
    \framebox{
        \vbox{\vspace{2mm}
        \hbox to 6.28in { {\bf 多尺度材料模拟与计算 \hfill #2} }
        \vspace{6mm}
        \hbox to 6.28in { {\Large \hfill #1 \hfill} }
        \vspace{6mm}
        \hbox to 6.28in { {\it 任课教师: {\rm #3} \hfill 姓名: {\rm #4}, 学号: {\rm #5}}}
        \vspace{2mm}}
    }
    \end{center}
    % \markboth{#4 -- #1}{#4 -- #1}
    \vspace*{4mm}
}


\newenvironment{solution}
{\color{blue} \paragraph{Solution.}}
{\newline \qed}

\begin{document}
%- Write your name and id here
\homework{Homework 2}{2023 秋}{孔令体、刘桂森}{张三}{XXXXXXXXXXXX}

\paragraph{Notice}
\begin{itemize}
    \item The submission email is: \textbf{zhangzhenyao@lamda.nju.edu.cn}.
    \item Please use the provided \LaTeX{} file as a template.
    \item If you are not familiar with \LaTeX{}, you can also use Word to generate a \textbf{PDF} file.
\end{itemize}

\paragraph{Problem 1: Inequalities}
~\\

\noindent
Let $x\in\mathbb{R}^n,y\in\mathbb{R}^n$, where $n$ is a positive integer. Let $\|\cdot\|$ denote the Euclidean norm.
\begin{enumerate}[a)]
	\item Prove the triangle inequality $\|x+y\|\leq\|x\|+\|y\|$.
	\item Prove $\|x+y\|^2\leq(1+\epsilon)\|x\|^2+(1+\frac{1}{\epsilon})\|y\|^2$ for any $\epsilon>0$.
\end{enumerate}
\emph{Hint}: You may need the Young's inequality for products, i.e. if $a$ and $b$ are nonnegative real numbers and $p$ and $q$ are real numbers greater than 1 such that $1/p+1/q=1$, then $ab\leq\frac{a^p}{p}+\frac{b^q}{q}$.

% \begin{solution}
% Write your answer here.
% \end{solution}

\paragraph{Problem 2: Convex sets}
~\\

\begin{enumerate}[a)]
    \item Show that a polyhedron $P=\{x\in \mathbb{R}^n: Ax \leq b, A\in \mathbb{R}^{m\times n}, b\in\mathbb{R}^m\}$ is convex.

    \item Show that if $S\subseteq  \mathbb{R}^n$ is convex, and $A \in \mathbb{R}^{m \times n}$,
  then $A(S) = \{ Ax : x \in S \}$, is convex.

    \item Show that if $S\subseteq  \mathbb{R}^m$ is convex, and $A \in \mathbb{R}^{m \times n}$,
  then $A^{-1}(S) = \{ x : Ax \in S \}$, is convex.
\end{enumerate}

% \begin{solution}
% Write your answer here.
% \end{solution}

\paragraph{Problem 3: Hyperplane}
~\\

\noindent
What is the distance between two parallel hyperplanes, i.e.,$\{x|a^\top x=b\}$ and $\{x|a^\top x=c\}$ ?

% \begin{solution}
% Write your answer here.
% \end{solution}

\paragraph{Problem 4: Examples}
~\\

\noindent
Let~$C \subseteq \mathbb{R}^n$~be the solution set of a quadratic inequality,
\begin{center}
$C = \{x \in \mathbb{R}^n|x^TAx + b^Tx + c \leqslant 0\}$
\end{center}
with~$A \in \mathbb{S}^n$, $b \in \mathbb{R}^n$, and~$c \in \mathbb{R}$.
\begin{enumerate}[a)]
    \item Show that~$C$~is convex if~$A \succeq 0$.

    \item Is the following statement true? The intersection of~$C$~and the hyperplane defined by~$g^Tx+h=0$~is convex if~$A+\lambda gg^T \succeq 0$~for some~$\lambda \in \mathbb{R}$.
\end{enumerate}


\paragraph{Problem 5: Generalized Inequalities}
~\\

\noindent
Let~$K^*$~be the dual cone of a convex cone K. Prove the following,
\begin{enumerate}[a)]
    \item $K^*$~is indeed a convex cone.

    \item $K_1 \subseteq K_2$~implies~$K^*_2 \subseteq K^*_1$.

\end{enumerate}
% \begin{solution}
% Write your answer here.
% \end{solution}

\end{document}
